\documentclass[10pt, compress]{beamer}

\usetheme{utopia}

\usepackage{booktabs}
\usepackage[scale=2]{ccicons}
\usepackage{graphicx}
\usepackage[style=numeric]{biblatex}
\usepackage{hyperref}

\usepgfplotslibrary{dateplot}
\addbibresource{bibliografia.bib}

\title{Posibles negocios derivados de una simulaci\'on de la General Paz }
\subtitle{Aplicaciones Computacionales en Negocios}
\date{12/09/2023}
\author{Tom\'as Curzio \\ Federico Giorgi \\ Gast\'on Loza Monta\~na  }
% \titlegraphic{\includegraphics[scale=0.2]{images/logo2018.pdf}} % Optional title page image, comment this line to remove it
\institute{Universidad Torcuato Di Tel}

\begin{document}

\maketitle

\section{El modelo}

\begin{frame}[fragile]

\frametitle{El lugar y el tiempo}
En este proyecto, buscamos simular un carril de la General Paz, mas precisamente, \href{https://www.google.com.ar/maps/dir/-34.6549026,-58.5273448/RN+A001,+Buenos+Aires/@-34.6307373,-58.5202617,14z}{\textcolor{blue}{el trayecto desde Liniers hasta Lugones}}. 

Para buscar datos representativos de los distintos picos de tr\'afico, simularemos una ma\~nana de d\'ia laboral, desde las 5:00 AM, con poca densidad de autos, pasando por las 7:30 - 8:30 AM, horario pico, para finalizar a las 10 AM, un horario mas calmo en cuanto a densidad de autos se refiere.

Para entender la simulaci\'on, debemos comprender quienes son nuestros agentes y como interaccionan entre s\'i.

\end{frame}

\begin{frame}[fragile]

\frametitle{Agentes}

Llamamos agentes a los conductores y su auto como un conjunto. Estos, poseen tres caracter\'isticas:

\begin{itemize}
\item $v_0$: Velocidad deseada -> Velocidad a la que le gustar\'ia ir al agente en caso de no estar restringido, representada en $m/s$. Tiene en cuenta la velocidad m\'axima de la Gral. Paz.
\item $\mathcal{T}$: Headway -> Distancia en segundos que desea tener el agente con el agente que est\'a delante. Lo que en escuelas de manejo nos recomiendan que sea 2''.
\item $l$: Longitud del veh\'iculo -> Longitud del auto del agente en metros.
\end{itemize}

\end{frame}

\begin{frame}[fragile]

\frametitle{Reglas de interacci\'on}
\begin{itemize}
\item Cada agente tiene una posici\'on $X_t$ que se modifica en cada fracci\'on de tiempo $t$, que en nuestro modelo se incrementa de a 1s.
\item A su vez, tienen una velocidad $V_t$.
\end{itemize}
Ambas se modifican segun las leyes f\'isicas de movimiento:
\begin{itemize}
\item $X_{t+1} = X_t + V_t * \Delta t$
\item $V_{t+1} = X_t + a_t * \Delta t$
\item $a_t = ?$ 
\end{itemize}

Los agentes toman decisiones en su aceleraci\'on, pudiendo frenar, mantenerla constante, o acelerar. Para determinar como lo hacen, utilizaremos un modelo conocido, llamado Intelligent Driver Model (IDM) \supercite{1}

\end{frame}

\begin{frame}[fragile]

\frametitle{Intelligent Driver Model}
El Intelligent Driver Model actualiza la posici\'on y velocidad como mencionamos anteriormente. El calculo de la aceleraci\'on es el siguiente, con $\alpha$ el n\'umero de agente y $\alpha - 1$ el agente de adelante.

  \begin{equation*}
    a_t = a (1 - (\frac{v_\alpha}{v_0})^2 - (\frac{s_0 + v\mathcal{T} + (\frac{v_\alpha (v_\alpha - v_{\alpha-1})}{2\sqrt{ab}})}{x_{\alpha-1} - x_\alpha - l_{\alpha-1}})^2
  \end{equation*}

Donde:
\begin{itemize}
\item $v_0$: La velocidad deseada del agente.
\item $s_0$: La minima distancia neta en metros (un auto no puede moverse si el de adelante esta a menos de $s_0$.
\item $\mathcal{T}$: El headway del agente.
\item $a$: La aceleraci\'on m\'axima del veh\'culo en $m/s^2$
\item $b$: La desaceleraci\'on maxima del veh\'iculo en $m/s^2$ (valor absoluto).
\end{itemize}

\end{frame}

\begin{frame}[fragile]

\frametitle{Los par\'ametros seleccionados}

\begin{itemize}

\item $\mathcal{T}$: Valores entre X y X para cada agente.
\item $v_0$: Valores entre X y X para cada agente, correlacionados con  $\mathcal{T}$
\item $l$: 4.3 metros para todos los agentes (longitud de un auto promedio)
\item $s_0$: 5 metros.
\item $a$: 2 $m/s^2$ -> Límites físicos charlados en clase.
\item $b$: 4 $m/s^2$ -> Límites físicos charlados en clase.
\item $\gamma$: Un nuevo par\'ametro que determina la proporci\'on de personas con waze o gmaps alertandolos de los radares. Utilizamos una proporcion de 0.6.

\end{itemize}

\end{frame}

\section{Resultados}

\begin{frame}
\frametitle{Cambio en los par\'ametros}

Al realizar el modelo, ten\'iamos en mente el objetivo de analizar el impacto de los radares, teniendo en cuenta que hay mucha gente hoy en d\'ia con aplicaciones que alertan de los mismos. Sin embargo, pudimos observar que los agentes que iban mas lento, generaban una fila de aquellos que querian ir mas r\'apido detr\'as de ellos, por lo que iban a velocidades lentas y los radares no generaban mucho impacto.

¿Acá puede ir una animación mostrando esto?


\end{frame}

\begin{frame}
\frametitle{Cambio en los par\'ametros}

Producto de una limitaci\'on del modelo, que es que tiene un solo carril, los agentes no pueden adelantarse. Para poder seguir en busca de observar datos que nos acerquen a nuestro objetivo, modificamos los parametros del \textit{headway} y \textit{velocidades deseadas}, intentando simular ahora el carril r\'apido, para poder captar de mejor manera el efecto de los radares, quedando de esta manera:

\begin{itemize}
\item $\mathcal{T}$: Valores entre X y X para cada agente.
\item $v_0$: Valores entre X y X para cada agente, correlacionados con  $\mathcal{T}$
\end{itemize}

\end{frame}

\begin{frame}
\frametitle{Resultados con los nuevos par\'ametros}
\end{frame}

\section{Posibles Negocios}

\begin{frame}
\frametitle{¿D\'onde poner radares?}
\end{frame}

\begin{frame}
\frametitle{Tradeoff: Recaudaci\'on vs Seguridad}
\end{frame}

\section{Conclusi\'on}

\begin{frame}
\frametitle{Conclusi\'on}
\end{frame}

\section{Referencias}

\begin{frame}[fragile]

\frametitle{Referencias}

\textbf{Referencias}:
\begin{itemize}
\item [{[1]}] Treiber, M., Hennecke, A., \& Helbing, D. (2000). Congested traffic states in empirical observations and microscopic simulations. Physical review E, 62(2), 1805.
\end{itemize}

\end{frame}

\plain{¿Preguntas?}

\end{document}
