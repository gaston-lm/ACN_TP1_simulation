\documentclass[10pt, compress]{beamer}

\usetheme{utopia}

\usepackage{booktabs}
\usepackage[scale=2]{ccicons}
\usepackage{graphicx}
\usepackage[style=numeric]{biblatex}

\usepgfplotslibrary{dateplot}
\addbibresource{bibliografia.bib}

\title{Posibles negocios derivados de una simulaci\'on de la General Paz }
\subtitle{Aplicaciones Computacionales en Negocios}
\date{12/09/2023}
\author{Tom\'as Curzio \\ Federico Giorgi \\ Gast\'on Loza Monta\~na  }
% \titlegraphic{\includegraphics[scale=0.2]{images/logo2018.pdf}} % Optional title page image, comment this line to remove it
\institute{Universidad Torcuato Di Tella}

\begin{document}

\maketitle

\section{El modelo}

\begin{frame}[fragile]

\frametitle{El lugar y el tiempo}
En este proyecto, buscamos simular un carril de la General Paz, mas precisamente, el trayecto desde Liniers hasta Lugones. 

Para buscar datos representativos de los distintos picos de tr\'afico, simularemos una ma\~nana de d\'ia laboral, desde las 5:00 AM, con poca densidad de autos, pasando por las 7:30 - 8:30 AM, horario pico, para finalizar a las 10 AM, un horario mas calmo en cuanto a densidad de autos se refiere.

Para entender la simulaci\'on, debemos comprender quienes son nuestros agentes y como interaccionan entre s\'i.

\end{frame}

\begin{frame}[fragile]

\frametitle{Agentes}

Llamamos agentes a los conductores y su auto como un conjunto. Estos, poseen tres caracter\'isticas:

\begin{itemize}
\item $v_0$: Velocidad deseada -> Velocidad a la que le gustar\'ia ir al agente en caso de no estar restringido, representada en $m/s$. Tiene en cuenta la velocidad m\'axima de la Gral. Paz.
\item $\mathcal{T}$: Headway -> Distancia en segundos que desea tener el agente con el agente que est\'a delante. Lo que en escuelas de manejo nos recomiendan que sea 2''.
\item $l$: Longitud del veh\'iculo -> Longitud del auto del agente en metros.
\end{itemize}

\end{frame}

\begin{frame}[fragile]

\frametitle{Reglas de interacci\'on}
\begin{itemize}
\item Cada agente tiene una posici\'on $X_t$ que se modifica en cada fracci\'on de tiempo $t$, que en nuestro modelo se incrementa de a 1s.
\item A su vez, tienen una velocidad $V_t$.
\end{itemize}
Ambas se modifican segun las leyes f\'isicas de movimiento:
\begin{itemize}
\item $X_{t+1} = X_t + V_t * \Delta t$
\item $V_{t+1} = X_t + a_t * \Delta t$
\item $a_t = ?$ 
\end{itemize}

Los agentes toman decisiones en su aceleraci\'on, pudiendo frenar, mantenerla constante, o acelerar. Para determinar como lo hacen, utilizaremos un modelo conocido, llamado Intelligent Driver Model (IDM) \supercite{1}

\end{frame}

\begin{frame}[fragile]

\frametitle{Intelligent Driver Model}
El Intelligent Driver Model actualiza la posici\'on y velocidad como mencionamos anteriormente. El calculo de la aceleraci\'on es el siguiente, con $\alpha$ el n\'umero de agente y $\alpha - 1$ el agente de adelante.

  \begin{equation*}
    a_t = a (1 - (\frac{v_\alpha}{v_0})^2 - (\frac{s_0 + v\mathcal{T} + (\frac{v_\alpha (v_\alpha - v_{\alpha-1}}{2\sqrt{ab}})}{x_{\alpha-1} - x_\alpha - l_{\alpha-1}})^2
  \end{equation*}

Donde:
\begin{itemize}
\item $v_0$: La velocidad deseada del agente.
\item $s_0$: La minima distancia neta en metros (un auto no puede moverse si el de adelante esta a menos de $s_0$.
\item $\mathcal{T}$: El headway del agente.
\item $a$: La aceleraci\'on m\'axima del veh\'culo en $m/s^2$
\item $b$: La desaceleraci\'on maxima del veh\'iculo en $m/s^2$ (valor absoluto).
\end{itemize}

\end{frame}

\begin{frame}[fragile]

\frametitle{Los par\'ametros seleccionados}


\end{frame}

\section{Resultados}

\section{Posibles Negocios}

\section{Conclusi\'on}

\section{Referencias}

\begin{frame}[fragile]

\frametitle{Referencias}

\textbf{Referencias}:
\begin{itemize}
\item [{[1]}] Treiber, M., Hennecke, A., \& Helbing, D. (2000). Congested traffic states in empirical observations and microscopic simulations. Physical review E, 62(2), 1805.
\end{itemize}

\end{frame}

\plain{¿Preguntas?}

\end{document}
